{\footnotesize
\begin{longtable}{| p{2.5cm} | p{1.5cm} | p{1.5cm} | p{1.5cm} | p{3cm} | p{3cm} |}\hline
	\textbf{Description} & \textbf{Likeli- hood} & \textbf{Impact (1-9)} & \textbf{Impor- tance} & \textbf{Preventive action} & \textbf{Remedialaction} \\
	\\\hline
	Data-loss (docs or code) 	& 3 & 7 & 21 
		& Continuous saving and version control. Distribute data among all group members. 
		& Try to retrieve data from computer/repository. Start from scratch, if necessary.
	\\\hline
	Network failure 			& 3 & 3 & 9  & 
		N/A & Switch to another network. Ask IDI for help, if necessary. 
	\\\hline
	Computer crash(es) & 4 & 7 & 28 & Continuous upload to repository & 
		Try to retrieve data. Worst case scenario: Retrieve most up-to-date data 
		from repository. Use computers from IDI (P15). 
	\\\hline
	Organization/ communication failure & 5 & 7 & 35  & Be watchful of correct 
		distribution of information. & Find out where communication failed and 
		restore the organization. 
	\\\hline
	Personnel absent due to sickness & 6 & 3 & 18 & Continuous upload to 
		repository. Be prepared to pitch in on others’ assignment, inform the 
		group that you are unable to meet on the day of sickness. \& Get update 
		and data from the person in question. Bring the person up-to-speed on 
		project, work done while he was away. & 
	\\\hline
	Great personal conflicts & 3 & 8 & 24 & Be prepared to withstand discussion 
		and criticism. Tell the other members of the group how you feel. Go to 
		the supervisor/ \& professor for advice if necessary. & Resolve the 
		issues with the help of a neutral part (supervisor/professor). 
	\\\hline
	Absent personnel & 7 & 3 & 21 & Point out the importance of attendance to 
		appointed hours. Inform ahead of time if leaving for vacation. See also 
		\emph{Personnel absent due to sickness}.  & Point out the importance of 
		attendance again. Notify student assistant if it becomes a problem. See 
		also \emph{Personnel absent due to sickness.}
	\\\hline
	Loss of personnel & 2 & 8 & 16 & Regular uploads of code and information 
		to repository & Notify supervisor of loss of personnel. Split work 
		assignments to personnel left in group.  
	\\\hline
	Incorrect system requirements & 3 & 8 & 24 & Be sure everyone has read and 
		understood the project requirements. Ask the customer if there any 
		uncertainties. & Resolve the incorrect requirements. Find out what went 
		wrong. Be confident that the other requirements are correct by 
		confirming with the customer.
	\\\hline
	Inexperienced with development technology & 8 & 3 & 24 & Be prepared to 
		search for information needed. Ask for help from supervisor if 
		necessary. Choose technology we are already comfortable with, if 
		possible. & Search and retrieve needed information. Get acquainted with 
		the technology needed. 
	\\\hline
	Misunderstand project & 2 & 9 & 18 & Make sure everyone has read and 
		understood the project goals, and that our goal(s) fit the customers 
		needs. Regular meetings with the customer for discussion regarding our 
		goals/customer needs. & Notify supervisor of grave errors/
		misunderstandings. Work out a rescue plan with the customer. 
	\\\hline
	Misunderstand subproblem & 2 & 8 & 16 & Make sure everyone involved has 
		read and understood the sub project problem/solution. & Correct the 
		mistakes and restart resolvement of subproblem. 
	\\\hline
	Error estimation of time needed & 5 & 5 & 25 & Be prepared for incorrect 
		time estimate, including error margins, avoid bursts. & Do bursts, 
		expand time schedules for work/work longer hours. 
	\\\hline
\caption{Risk assesment list}
\label{fig:risklist}
\end{longtable}
}
