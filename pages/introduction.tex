\section{Introduction}

\subsection{The course}

The main goals in this course is to experience and learn how to work on a development project as a team. In addition, the team has to answer to a customer, as software development companies often do, which stands out from other projects in the past. This is an advanced course and it is expected of us to use knowledge obtained from previous courses, especially development courses such as Informatics Project I and the collaborative System Development project. 

The group has an appointed guidance counselor as well as a customer. The counselor is there to help with questions regarding the project management in general, and push the group to reflect on its decisions and review the work done. Status reports will be delivered regularly, so the counselor can stay up-to-date with the work in the group. 

During the course, several project reports are scheduled for delivery; the preliminary project report, the mid-term project report and the final project report. Working on and delivering these reports help us plan and develop the project, and we will receive feedback from our counselor. The grading of the project will take the final project report into consideration, as well as the final product and the customer’s satisfaction?

\subsection{The team}

The team consists of six guys studying informatics at NTNU:\\
\textbf{Stian Aurheim}
\begin{itemize} \setlength{\itemsep}{0cm}\setlength{\parskip}{0cm}%
	\item Third year bachelor in informatics
	\item Main experience in Java. Some experience in Python, PHP, HTML, JavaScript.
\end{itemize} 
\textbf{Jens Even Berg Blomsøy}
\begin{itemize} \setlength{\itemsep}{0cm}\setlength{\parskip}{0cm}%
	\item Third year bachelor in Informatics. 
	\item Programming languages worked with: Java, Python.
	\item Main knowledge in System Development, system architecture and system documentation.
\end{itemize}
\textbf{Jørgen Foss Eri}
\begin{itemize} \setlength{\itemsep}{0cm}\setlength{\parskip}{0cm}%
	\item Third year bachelor in Informatics. 
	\item Has experience with Java, Python, JavaScript/HTML5/CSS3 and general web development.
\end{itemize}
\textbf{Jim Frode Hoff}
\begin{itemize} \setlength{\itemsep}{0cm}\setlength{\parskip}{0cm}%
	\item Third year bachelor in Informatics
	\item Programming languages worked with: Java, Python, PHP, JavaScript and general web development
\end{itemize}
\textbf{Adrian Arne Skogvold}
\begin{itemize} \setlength{\itemsep}{0cm}\setlength{\parskip}{0cm}%
	\item Third year bachelor in informatics
	\item Programming languages worked with: Java, C\#, Oz, Actionscript
\end{itemize}
\textbf{Sindre Svendsrud}
\begin{itemize} \setlength{\itemsep}{0cm}\setlength{\parskip}{0cm}%
	\item Third year bachelor in Informatics. 
	\item Experience with Java, Python and C++.
\end{itemize}

\subsection{Problem description}
The customer has developed a paper prototype of a board game called Don’t Panic. The game is supposed to help crisis workers make the right decisions during a crisis. It is a turn based, collaborative multiplayer game where the players take actions to stop the city from panicking. After the game is finished, an expert is supposed to review the actions with the players to evaluate whether their choices were sound. 

Our customer wants us to make an electronic version of the board game depicted in Figure ~\ref{fig:paperPrototype}. The electronic board game should maintain the social aspect (both physical and verbal) of a regular board game. In addition, we are adding a replay function, to make it easy for the expert to review the game with the different players. The physical version of the board game takes a lot of work setting up and maintaining, since there is a lot of work moving the pieces and updating the panic levels. The electronic version will automate all of this.

% TODO: fix floating
\begin{figure}[here]
  \centering
    \includegraphics[width=1.0\textwidth]{img/paper_prototype}
  \caption{Picture of the paper prototype, complete with colored zones, circular panic gauges, nodes, information centers, barricades and players.} \label{fig:paperPrototype}
\end{figure}

\subsection{Constraints}
\begin{itemize} \setlength{\itemsep}{0cm}\setlength{\parskip}{0cm}%
	\item Only a couple of us has any past experience with HTML5 and javascript
\item The game is to be done within one semester (21 January - 27 May)
\end{itemize}

\subsection{Customer and supervisor}

Our customer for this project is Ines Di Loreto, a researcher at the Department of Computer \& Information Science, NTNU. The supervisor we have been assigned for this project is Mohsen Anvaari, PhD candidate at the same department






