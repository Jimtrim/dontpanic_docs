\chapter{Requirements}
\section{Functional requirements}

\begin{itemize} \setlength{\itemsep}{0cm}\setlength{\parskip}{0cm}
	\item The game interface should provide experts with the ability to efficiently manage and control the players.
	\item The players must be able to interact with the client using an I/O device.
	\item The game interface should have elements for managing game sessions.
	\item All game objects that exist longer than a single task, must be created and kept in storage. An object is anything from the players’ data to an invisible trigger object.
	\item Game rules should be accepted as parameters in input. These parameters should be set by the experts through the management client. 
	\item The expert should be able to follow a game session and comment on the tracking of the game, but should not be able to participate.
	\item The expert should have the ability to effectively monitor the server.
\end{itemize}
\subsection{Use cases}
The use cases are mainly based on the functional requirements of the game and are a graphical representation of the users’ interactions with the board game. They document all the different ways in which the user can interact with the game. 
\\
A detailed set of use case diagrams and textual use cases are provided below.\\


% START USE-CASE TABLES
\begin{table}[H]
\begin{tabular}{|l|l|} \hline
	\textbf{ID} & \textbf{01}\\ \hline
	Name & Login Player\\ \hline
	Goal & To be connected to the server\\ \hline
	Actors & Player, server\\ \hline
	Start requirements & None\\ \hline
	End requirements & - The player gets logged in.\\
					 & - The game is displayed.\\ \hline
	Case & - The player gets prompted with the login window. \\
		 	& - The players gives login-info.\\
			& - The player is now logged in.\\ \hline
	Alternative Case & Wrong password\\ \hline
	Previous Use Case & None\\ \hline
	Spawned Use Case & 05\\ \hline
\end{tabular}
\caption{Use Case: Login player}
\label{fig:usecase01table}
\end{table}


\begin{table}[H]
\begin{tabular}{|l|l|} \hline
	\textbf{ID} & \textbf{02}\\ \hline
	Name & Login Expert\\ \hline
	Goal & To be connected to the server\\ \hline
	Actors & Expert, server\\ \hline
	Start requirements & None\\ \hline
	End requirements & - The expert gets logged in.\\
					 & - The expert view is displayed.\\ \hline
	Case & - The expert gets prompted with the login window. \\
		 	& - The expert gives login-info.\\
			& - The expert is now logged in.\\ \hline
	Alternative Case & Wrong password\\ \hline
	Previous Use Case & None\\ \hline
	Spawned Use Case & 03\\ \hline
\end{tabular}
\caption{Use Case: Login expert}
\label{fig:usecase02table}
\end{table}

\begin{table}[H]
\begin{tabular}{|l|l|} \hline
	\textbf{ID} & \textbf{03}\\ \hline
	Name & Game Setup\\ \hline
	Goal & To create a successful game session\\ \hline
	Actors & Expert, server\\ \hline
	Start requirements & The expert is logged in\\ \hline
	End requirements & The expert is able to create a game setup \\ 
						& The expert is able to save the game setup\\ \hline
	Case & The expert creates the appropriate map for the game.\\
			& The expert adds the wanted board pieces.\\
			& The expert manages the zones, the number of people and panic levels.\\
			& The expert manages the cards, adds the wanted cards to the game.\\
			& The expert adds the wanted number of players to the game.\\
			& The expert assigns roles to each player. \\
			& The expert sets the starting point for each player.\\ \hline
	Alternative Case & None \\ \hline
	Previous Use Case & 02\\ \hline
	Spawned Use Case & 04, 05\\ \hline
\end{tabular}
\caption{Use Case: Game Setup}
\label{fig:usecase03table}
\end{table}

\begin{table}[H]
\begin{tabular}{|l|l|}
\hline
	\textbf{ID} & \textbf{04}\\ \hline
	Name & Watcher\\ \hline
	Goal & To get a non player version of the game\\ \hline
	Actors & Expert, server\\ \hline
	Start requirements & The expert is logged in, a game is running \\ \hline
	End requirements & The expert is able to watch the wanted game\\ \hline
	Case & The expert selects the game to watch from a list of games\\
		 & The server provides a game window in which the expert is not\\ 
		 & participating as a player\\ \hline
	Alternative Case & None \\ \hline
	Previous Use Case & 02\\ \hline
	Spawned Use Case & None\\ \hline
\end{tabular}
\caption{Use Case: Watcher}
\label{fig:usecase04table}
\end{table}

\begin{table}[H]
\begin{tabular}{|l|l|}
\hline
	\textbf{ID} & \textbf{05}\\ \hline
	Name & Join Game\\ \hline
	Goal & To successfully join a starting game\\ \hline
	Actors & User, game session, server \\ \hline
	Start requirements & A game has been created by the exper\\ \hline
	End requirements & A user is able to join the appropriate game\\ \hline
	Case & The user clicks on join options for the game\\
			& The game asks for user info \\
			& The user joins the game \\ \hline
	Alternative Case & The user gives incorrect info and is not added to the game \\ \hline
	Previous Use Case & 03 \\ \hline
	Spawned Use Case & 06, 07\\ \hline
\end{tabular}
\caption{Use Case: Join Game}
\label{fig:usecase05table}
\end{table}



\begin{table}[H]
\begin{tabular}{|l|l|}
\hline
	\textbf{ID} & \textbf{06}\\ \hline
	Name & Move game pieces \\ \hline
	Goal & To move a game piece to a wanted location \\ \hline
	Actors & Player, game board, game session, server \\ \hline
	Start requirements & The player has joined a game \\
				& The player in question has the turn \\ \hline
	End requirements & The player is able to move the selected piece to the wanted position. \\ \hline
	Case & The player uses the game pad to select the wanted object. \\
		& The player drags the object to the wanted location. \\
		& The game board is updated. \\ \hline
	Alternative Case & The player selects an immovable object \\
				& The player moves the object to an unobtainable location\\ \hline
	Previous Use Case & 06 \\ \hline
	Spawned Use Case & None\\ \hline
\end{tabular}
\caption{Use Case: Move game pieces}
\label{fig:usecase06table}
\end{table}

\begin{table}[H]
\begin{tabular}{|l|l|}
\hline
	\textbf{ID} & \textbf{07}\\ \hline
	Name & Use information cards\\ \hline
	Goal & To use an information card to affect the board\\ \hline
	Actors & Player, Game Board, Game Session, Server\\ \hline
	Start requirements & The expert has created a game\\
				& The player is logged in\\
				& The player is part of a game\\
				& The player has an information card \\ \hline
	End requirements & The card effect is carried out on the board\\
				& The player does not have the used information card \\ \hline
	Case & The player selects the information card\\
		& The information card effect is carried out on the board\\
		& The player loses his information card \\ \hline
	Alternative Case & None \\ \hline
	Previous Use Case & 05\\ \hline
	Spawned Use Case & None\\ \hline
\end{tabular}
\caption{Use Case: Use information cards}
\label{fig:usecase07table}
\end{table}


\begin{table}[H]
\begin{tabular}{|l|l|}
\hline
	\textbf{ID} & \textbf{08}\\ \hline
	Name & Use player action \\ \hline
	Goal & The player uses an action and the game board is updated\\ \hline
	Actors & Player, Game Board, Game Session, Server \\ \hline
	Start requirements & The user is logged in \\
				& The user is a player in the game\\
				& A game is in action\\ \hline
	End requirements & The player uses an action \\
				& The effect is updated on the board \\ \hline
	Case & The player selects an action with the gamepad\\
		& The player selects a zone or a node\\
		& The action is used on the target\\
		& The game board is updated\\ \hline
	Alternative Case & None\\ \hline
	Previous Use Case & 05\\ \hline
	Spawned Use Case & None\\ \hline
\end{tabular}
\caption{Use Case: Login expert}
\label{fig:usecase08table}
\end{table}
% END USE-CASE TABLES



\section{Non functional requirements} 


All non functional requirements comply with the definitions as stated in the ISO 25010 standard (replacing ISO 9126). Only relevant requirements are mentioned in this report.

\subsection{Quality in use}

\subsection{Efficiency}
Like regular board games, actions should not be difficult to execute. The players are working against the clock (the panic increase timer). Hence, when designing the user interface, one of the aims should be to minimize the number of clicks required.

\subsection{Context coverage}
The system should be flexible enough to accommodate individual experts’ preferences and needs in their simulations. By relying on the settings given by the expert through the expert interface form, the best possible flexibility can be ensured.

\subsection{Product quality}

\emph{1: Functional suitability}\\
Functional completeness should be achieved to include the core functionality 
of the board game, as well as the functionality specific to the electronic 
version, like the expert interface and panic- and people management. 
\\
Core functions must be without game-breaking bugs to ensure functional 
correctness.
\\\newline
\emph{2: Operability}\\
Ease of use is considered important, as the users should spend time playing the 
game and learn how to manage panic, rather than how to operate the game. 
\\
By exploiting recognisability from classic board games, a lot of interaction 
can be made intuitive, given that most people already know how to play board 
games. 
\\
Users of the game will most likely not be as proficient with computers as 
“gamers” in general. Therefore, it would be a good idea to make the game 
accessible without having to install anything other than an internet browser.
\\\newline
\emph{3: Transferability}\\
The client should be usable on as many platforms as possible (Mac, Windows, 
Linux, Mobile platforms), and in the best possible case be able to interact 
with devices such as Arduino. HTML5 with node.js was chosen for this reason, as 
it can run on nearly any device without the need for time consuming 
installation procedures.


\subsection{Technical requirements}
These requirements have been copied from the “Don’t Panic” specifications 
provided by the customer.\\

Don’t Panic DPS has to meet to the following requirements:\\
- All interaction between the server and client SHOULD be performed using well 
documented protocols and standard protocols.\\
- The DPS Game rules SHOULD be platform independent. Consequently, high level 
languages such as Java, Processing, Python COULD be considered as good 
candidates.\\
- The overall architecture SHOULD be scalable to run multiple Game sessions in 
parallel without decreasing the quality of already running games sessions.\\
-Already existing frameworks for game development for such as Unity, Microsoft 
XNA Game Studio, or management tools such as RedMine COULD be used as platforms 
to help speeding up the development of the game. The choice should be driven by 
a framework comparison analysis considering both technical requirements and 
already existing skills/experience among group participants.






