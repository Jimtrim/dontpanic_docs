\chapter{Design and architecture}

\section{User interface}
% TODO: image 
Fig x. The user interface of the game, complete with a board, pieces, cards and information tables







\subsection{User interface design}

The customer had already designed a physical board game of Don’t Panic, which formed the basis for designing the electronic version. In the earliest stages of the designing Balsamiq Mockups was used, an online tool for interface designing.\\
\\
Mockups is designed to be an easy and efficient tool used in the early stages of interface designing, and it can be used to generate click-through prototypes for interfaces. Through myBalsamiq it can also be used as a collaborative tool, supporting project-based collaboration and real-time changes.\\
\\ % TODO: image
Fig X: Early design of the board, cards and the expert interface  

\subsection{Realisation of the user interface}

% TODO: image
Fig x: The canvas\\
\\
The main part of the interface is the HTML5 canvas. The board and pieces of the game are all drawn within this canvas using JavaScript. The players are able to control their respective player pieces by dragging them on the board from one node to another using the mouse, like one would do using the hands when playing the physical version of the board game. One of the main goals of the project was to preserve the physical interaction of playing a board game in the electronic version, so it was decided that implementing a drag-and-drop functionality for the player pieces would be a good idea. \\
\\
The panic levels for each zone are printed inside the zones. This was very cumbersome with the physical version, as the players were forced to update all the zones manually each time the timer counted down, as well as when event cards and information affected the zones. This is now handled automatically on the server. \\
\\ % TODO: Image
Fig x: The head table\\
\\
An information table is placed above the canvas. This table contains the button the players would use when ending their turns, as well as buttons for placing information centers and roadblocks on the nodes they are located. The table also contains information about whose turn it is, how many turns have passed, how many actions the active player has left and a timer which counts down to the next increase in panic. \\
\\
On each side of the canvas there are sidebar rows. Players have their own row, which is highlighted when it is their turn. The rows contain information on players, as well as their information cards. Since this is a collaborative game, all the cards should be visible to the players so they can openly discuss how the cards should be used, as well as trade cards between themselves. This preserves the verbal interaction dynamics of a physical board game. The cards are implemented as buttons with text corresponding to their effect in the game. \\
\\ %TODO: image
Fig x: Sidebar rows



\section{Client/Server architecture}
A client-server model was chosen as the architectural pattern. This was highly desired by the customer, as they wanted different clients
to work with the server. Therefore using a different architecture was never considered as an option.

\subsection{Initial suggestion for a high level architecture from the customer}
% TODO: image
Se figur over: Skal det være apostrof etter Administration?

\subsection{High level system architecture}
% TODO: image

\subsection{Object diagram}
% TODO: image


\subsection{ER diagram}
% TODO: image

\subsection{Object hierarchy}
This chart is intended to show a high level view of the hierarchy, and how the objects are connected together to avoid loops. 








