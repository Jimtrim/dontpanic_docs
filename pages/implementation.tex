\chapter{Implementation}

This is an overview of the implementation process and how the product evolved throughout the project. In addition to this chapter, the status reports can be found in the appendix. \\

\section{Iteration 1}

Week 5 - 8\\
\newline
In this period, most of the time was spent on research, learning, and gathering and clarifying the requirements. The group lacked extensive experience with JavaScript and Node.js, so it was necessary to spend the first weeks on getting to know these technologies. Translating the requirements of a board game into a version that would work in an electronic format also took a significant amount of time in those weeks.\\
\newline
Week 8 - 9\\
\newline
The first version of the user interface, and a barebones server was implemented. Algorithms for constructing nodes, zones and paths, as well as proper listening functions for selecting objects were developed.\\

\begin{figure}[H]
  \centering
    \includegraphics[width=1.0\textwidth]{img/mockups.png}
  \caption{User interface, ' Early mockup of the board, cards and the expert interface'} 
  \label{fig:mockups}
\end{figure}

%Initial interface
\begin{figure}[H]
  \centering
    \includegraphics[width=1.0\textwidth]{img/canvas.png}
  \caption{User interface, 'First Iteraton canvas'} 
  \label{fig:canvas}
\end{figure}

\section{Iteration 2}



Week 10 - 14\\
\newline
Early in iteration 2, the server was in focus, and was populated with most of the core game functionality (FR5) such as moving players, decreasing panic, info cards, events, and timer. Sidebars for player information was added, as well as a status bar showing relevant information with buttons to execute actions on selected items. In addition a simple database connection, and CSS was implemented.\\
\\
Later on in this iteration functionality for adding and removing roadblocks and information centers were added. 
% FIG better interface
\begin{figure}[H]
  \centering
    \includegraphics[width=1.0\textwidth]{img/earlyVersion.png}
  \caption{User interface, ' User interface with players, cards and status bar'} 
  \label{fig:earlyversion}
\end{figure}
At the end of this iteration new graphics were added, the zones was now images depending on what type of zone it was instead of just colors.

%New graphics interface
\begin{figure}[H]
  \centering
    \includegraphics[width=1.0\textwidth]{img/newgraphics.jpg}
  \caption{User interface, 'Canvas with images instead of colors'} 
  \label{fig:newgraphics}
\end{figure}

\section{Iteration 3}

Week 15 - 18\\
\newline
The third iteration was focused on the functionality outside the game. The expert interface was created, which creates game templates and maps that can be loaded into the game. The website got an index page, and general website functionality so users can navigate between the pages easily.\\
\\
In the end of this iteration the game templates that the expert interface made was successfully stored in the database and could be retrieved by the server. Therefore users could now create game templates and play different game templates instead of just the standard template that had been used until now.\\
\begin{figure}[H]
  \centering
    \includegraphics[width=1.0\textwidth]{img/gamefinal.png}
  \caption{Game Interface, 'Final version'} 
  \label{fig:gamefinal}
\end{figure}
% FIG expert interface

\begin{figure}[H]
  \centering
    \includegraphics[width=1.0\textwidth]{img/ExpertInterfaceForms.png}
  \caption{Expert Interface, 'Final version'} 
  \label{fig:EcpertInterfaceForm}
\end{figure}
% FIG expert interface


\begin{figure}[H]
  \centering
    \includegraphics[width=1.0\textwidth]{img/ExpertInterfaceCanvas.png}
  \caption{Expert Interface Map Creation, 'Final version'} 
  \label{fig:EcpertInterfaceCanvas}
\end{figure}
% FIG expert interface

\section{Iteration 4}

Week 18 - 19\\
\newline
The last few weeks of coding were spent on stabilizing the codebase, as well as adding more complicated features that had been worked on for some time, like the ability to watch replays of games that have been played, the Game Master interface which enables an expert to watch a game in progress and take control of players, and the ability to translate the game and website into other languages.\\


%FIG replay, index in norwegian and english

\begin{figure}[H]
  \centering
    \includegraphics[width=1.0\textwidth]{img/indexen.png}
  \caption{Index, ' Index page in english'} 
  \label{fig:indexen}
\end{figure}

\begin{figure}[H]
  \centering
    \includegraphics[width=1.0\textwidth]{img/indexno.png}
  \caption{Index, ' Index page in norwegian'} 
  \label{fig:indexno}
\end{figure}


