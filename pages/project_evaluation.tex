\chapter{Project evaluation}

\section{Team}

\subsection*{Task assignment}
To begin with the task assignment did not work well. The different people in 
the group did not know what the others were doing, and therefore some tasks were done twice.
After a team leader was chosen, the task assignment were much better and the group worked more efficiently. 

\subsection*{Roles assignment}
The roles assignment worked well. None of the members in the group were dissatisfied with their role, and each role
did their part.

\subsection\subsection{Evaluation of the different roles}
TODO


\section{Process model}

How did lean waterfall work out?

\section{Development language}

How was javascript to work with?


\section{Customer evaluation}

TODO

\section{Difficulties within the project}

TODO

\section{Lessons learned}

Early in the project there were daily meetings. Later on the need for daily meetings was regarded as unnecessary as the team was able to work individually with the appointed tasks. Therefore, the number of meetings was reduced to three days a week (Mondays, Wednesdays and Fridays). On Tuesdays and Thursdays the team members worked at home. Mondays were used as a sort of “kick-off”-day for the week and were used to plan specific tasks that were to be done during the week. This increased the efficiency.\\
\\
During week 9 the group member responsible for sending status reports to the customer and supervisor was on vacation. Because of that, no status report was sent that week. The lesson learned from this is that communication within the team is crucial for the completion of all tasks in time.\\
\\
In the start of the project, there were little communication within the group. No standards were set
for coding purposes and later on when the database names did not match the names in the program, there were some unnescessary
problems. This was solved by setting code standards and better communication between the different 
parts of the team. 
