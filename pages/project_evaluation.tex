\chapter{Project evaluation}

\section{Team}

\subsection*{Task assignment}

In the early stages of development, the focus was was set on learning JavaScript and getting to know Node.js. Since only a few of the developers had experience with Javascript, this was critical to further planning. Communication within the group was at a minimum, with each member doing his own research. In hindsight, this process could have been improved by agreeing on certain aspects for each member to go deeper into, and then present their findings to the rest of the group. After a week of reading individually, the group discussed possible assignment of roles and tasks. With no appointed leader at this point, it was a question of what each different member would preferably do. Assignment of task went well, no member felt he was assigned to a task he strongly disliked.


\subsection*{Role assignment}

The group appointed roles based on the larger tasks within the project, as the expert interface, game logic, database and so fourth. As was stated before, no member was deeply dissatisfied with their roles. Though the role of Sifteo manager which came much later in the project, was somewhat harder to assign, due to the already weighing work load. The appointment of team leader was done initially to one member at the time of the other role assignments, but later changed. 

\subsubsection{Evaluation of the different roles}
\begin{itemize} \setlength{\itemsep}{0cm}\setlength{\parskip}{0cm}
	\item Customer/supervisor contact \\
	The contact role has had very important tasks for the group, making sure status report has been sent to the supervisor,  had contact with the customer and scheduling meetings. Sometimes it was difficult to predict how much time a task would take. Therefore sending weekly status reports with the next weeks assignments was not easy, and did not always match what assignments we actually did the next week. 
	\item Server manager\\
Other than the usual coding difficulties the server was completed without much hassle. The one responsible for the server has been the goto guy for questions relating to this part of the game.
	\item \LaTeX configuration manager\\
The role of \LaTeX manager was introduced right after the midterm report was delivered. As mentioned earlier the group used Google Docs to write the report, until it started acting up, due to the length of the document. The group have had small difficulties working with \LaTeX because only a few of us has previous experience with the program. The \LaTeX manager is the one with the most \LaTeX experience and has been our goto guy, from small to larger problems.

	\item Client manager\\
The client managers has had a larger workload containing coding, only one of these has had experience with Javascript before, and has been prompted with difficult questions most days.
	\item Expert interface manager\\
The expert interface manager has had a great workload containing coding, being the only one working on this separate part of the game. The expert interface manager has had small to no problems completing his tasks.
	\item Team manager\\
The team manager has not been strongly enforced, there has not been much need for an team manager to tell each member what to do at all times. The team manager has been important in the starting stages of the project, where some had to take actions concretely describing what needed to be done. 
	\item Test manager\\
The test manager has been responsible for writing tests and making sure the test would be completed. Testing of the more finished product with both customer and independent people has been important. Testing in the last stages has been needed to tweak and change small functions and views. To get confirmation that the product was up to par functionally and quality wise has been great, and the test manager has been a great help enforcing this.
	\item Database manager\\
Having one person fully responsible for the database was much needed. When the features within the game went from being hard coded to dynamic, the database manager was the one supervising this transition. 
	\item Documentation manager\\
The documentation manager has been the one finding out what needed to be written within the report, and has been of great use. He made several TODO lists and worked to make things easier to work with. The documentation manager has also been the one to find faults and making sure the whole report has been proofread.
\end{itemize}


\section{Process model}

The modified waterfall model has worked fine. There have been some deviations to the model though. New requirements were
proposed by the customer later in the project, when the design phase was largely completed, resulting in the process model not being followed
to the letter. This was not a major, issue, however. The design and implementation part worked very well, since the 
team maintained contact and frequent meetings and correspondence with the customer. If some features were unsatisfying, the design process would 
start over. Even though this was possible, only small changes needed to be made. Being able to push forward to completion, without spending excessive time on perfecting every part before the next feature could be introduced, has been good for the development and progress of the game. 

\subsection{Project Planning}

As specified by the process model, most of the planning happened in the early phases. Although initial plans were provided by the customer, they needed significant rewriting to fit the workflow dictated by node.js and JavaScript, and to conform to the additional requirements that were added during the course of the project. With the limited experience the group had regarding JavaScript, more detailed planning would probably have been wasted as the product evolved and more knowledge waas gained. In all, the planning phase was shorter than normal, but necessary and useful.


\section{Development language}

The chosen development language Javascript has been rewarding to work with. Being the most used language on the web, expanding our knowledge and skill with JavaScript has expanded our understanding of websites and programming in general. Compared to languages such as Java, which has strong typing, JavaScript is loosely typed, meaning the variable types are not explicitly stated and can be changed at runtime. It is also dynamic and has first-class function, which allows a lot of flexibility and means objects can be extended and modified at runtime, functions can be passed as arguments to other functions, and the program does not require compiling.\\
\newline
Although JavaScript introduces a lot of new, useful concepts, understanding how to use them in practice was a challenge at first. Examples of this included asynchronous functions, which are used a lot in Node.js and especially when communicating with a database. Understanding the flow of such programs required a lot of trial and error, but resulted in better understanding and more effective functions, compared to blocking functions for the same tasks.\\


\section{Group interaction}

In comparison to earlier project, group interaction has overall been better than before, according to the group members. It started out slow, but got progressively better as the members got to know each other. This resulted in a higher coding efficiency, as the members were quick to help one another when someone ran into problems. For the most part, it was easy to know where the individual members were when they did not attend a meeting as a result of using a facebook group.

\section{Customer interaction}

Having to deal with a customer for this project has been a great experience and opportunity to learn the dynamic of such a relationship. Meetings with the customer has happened almost weekly, this to show the newest progress and features in person. The discussions during the meetings have been about the new features and any objections the customer may have had. 

\section{Supervisor interactions}
Having a supervisor to lean on has been a great help when it comes to the report. Getting feedback throughout the course has been crucial to create the best possible documentations for the game. The supervisor has helped with questions about our development process and especially functional requirements. Meetings with the supervisor has happened every other week, the topic of discussion has been the problems the team was facing and the planned work for the next period. \\
\\
The group find it a bit odd that no feedback was given on the report after 19th april. The chapters Project evaluation, Testing and Final product were all written after the program was finished and no feedback were given on these chapters.

\section{Final product}

The game has been a major challenge, due to the large amount of requested features from the customer. Although the number of requested features was larger than what the team had time to implement, they were sorted by priority in talks with the customer, and it was never expected that all the features would be fully implemented. During the course of the project, the customer continued to request additional features, some of which were added on the fly, while others were put on a waiting list and not implemented due to time constraints.\\
\newline
The group is satisfied with the final product, having been able to implement many new features close to the end, and making the game as stable as was feasible. The customer expressed satisfaction with the product after testing it, even though the test might have been placed a bit too early when the game was not as stable as the final version. The final product functions as intended, even without the unfinished features.\\

\section{Lessons learned}

Early in the project there were daily meetings. Later on, the need for daily meetings was regarded as unnecessary, as the team was able to work individually with the appointed tasks; therefore, the number of meetings was reduced to three days a week (Mondays, Wednesdays and Fridays). On Tuesdays and Thursdays the team members worked at home. Mondays were used as a sort of “kick-off”-day for the week and were used to plan specific tasks that were to be done during the week. This increased the efficiency.\\
\newline
During week 9 the group member responsible for sending status reports to the customer and supervisor was on vacation. Because of that, no status report was sent that week. The lesson learned from this is that communication within the team is crucial for the completion of all tasks in time.\\
\newline
In the start of the project, there was limited communication within the group. No standards were set
for coding purposes and later on when the database names did not match the names in the program, there were some unnecessary problems. This was solved by setting code standards and better communication between the different parts of the team. \\
\newline
Dealing directly with a customer has been a new experience for many of the group members. Having to answer to an external entity, rather than a textual task list or the group itself, has been eye opening and certainly a lesson learned. Dealing with changing and increasing requirements, weekly status reports and presentations for the customer has been more in line with what real projects are assumed to be like, contrary to earlier projects and assignments.
\newline
Earlier testing could have been helpful to keep the game more stable. Both planning and writing unit tests could have both increased understanding of the system, as well as pointed out which parts of the system was lacking in stability and documentation. 
\newline
Documentation before and during coding has not been done in an optimal way. There was several instances of clashing conventions and confusion regarding how certain functions worked, and how data should be formatted to work in all parts of the system. This was largely overcome through good communication and frequent group meetings, where the developers working on the individual parts helped out the others one on one when issues occurred. Yet, this could have been more effective if documentation was more detailed.





