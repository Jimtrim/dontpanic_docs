\chapter{Project evaluation}

\section{Team}

\subsection*{Task assignment}

When the group first started our development the group had no structure of dividing tasks among us. Since only a few of us had experience with Javascript, the group were first of all trying to acquire some skill within this language. The group were not working as a group should but only a vague understanding of what each other were doing. After some time working on our own and acquiring skills the group had a discussions about the assignment of roles and tasks. The group discussed the different big tasks within the project and appointed roles based on them. It was a questions of what each different member would preferably do. Assignment of task went well, no member felt he was put on a task he strongly disliked.

\subsection*{Role assignment}

The group appointed roles based on the big tasks within the project, expert interface, game, database and so fourth. As was stated before no member was deeply dissatisfied with their roles. Though the role of Sifteo manager which came much later in the project, was somewhat harder to assign, due to the already weighing work load. The appointment of team leader was done initially to one member at the time of the other role assignments, but later changed. 

\subsubsection{Evaluation of the different roles}
\begin{itemize} \setlength{\itemsep}{0cm}\setlength{\parskip}{0cm}
	\item Customer/supervisor contact \\
	The contact role has had very important tasks for the group, making sure status report has been sent to the supervisor,  had contact with the customer and scheduling meetings. Some of the tasks has been hard to follow such as sending status reports due to the fact that next weeks assignments was not always clear. 
	\item Server manager\\
Other than the usual coding difficulties the server was completed without much hassle. And the one responsible for the server has been the goto guy for questions relating this part of the game.
	\item LaTeX configuration manager\\
The role of LaTeX manager was introduced right after the midterm report was delivered. As mentioned earlier the group used Google Docs to write the report, until it started acting up, due to the length of the document. The group have had small difficulties working with LaTeX because only a few of us has previous experience with the program. The LaTeX manager is the one with the most LaTeX experience and has been our goto guy, from small to larger problems.

	\item Client manager\\
The client managers has had a larger workload containing coding, only one of these has had experience with Javascript before, and has been prompted with difficult questions most days.
	\item Expert interface manager\\
The expert interface manager has had a great workload containing coding, being the only one working on this separate part of the game. The expert interface manager has had small to no problems completing his tasks.
	\item Team manager\\
The team manager has not been strongly enforced, there has not been much need for an team manager to tell each member what to do at all times. The team manager has been important in the starting stages of the project, where some had to take actions concretely describing what needed to be done. 
	\item Test manager- Jens Even Berg Blomsøy
The test manager has been responsible for writings tests and making sure the test would be completed. Testing of the more finished product with both customer and independent people has been important. Testing in these last stages has been needed to tweak and change small functions and views. To get confirmation that the product was up to the thought functionality and quality has bean great, and the test manager has been a great help enforcing this.
	\item Database manager - Jens Even Berg Blomsøy
Having one fully responsible for the database has been much needed, when the features within the game went from being hard coded to dynamic, the database manager was the one supervising this transition. 
	\item Documentation manager - Sindre Svendsrud
The documentation manager has been the one finding out what needed to be written within the report, and has been of great use. Making TODO lists and making things easier to work with. The documentations manager has also been the one to find faults and making sure the whole report has been proofread.
\end{itemize}


\section{Process model}

Our process model the modified waterfall model has worked fine. The team have had some  deviations to the model though. New requirements was proposed by the customer later in the project when the design had already started. So the process model has not been followed completely, this however has not been a problem. The design  and implementation part has worked very well for us, since the team have had good contact and regular meetings with the customer. If some features were unsatisfying, the design process could start over again. Even though this was possible only small changes has had to be made. The customer has for the very most part been satisfied with the new features showed each time. Having the verification part at the end of our process has suited us great. Being able to push forward to completion, without every part having to be thoroughly tested before a next feature could be introduced, has been good for the development and progress of the game. \\

\subsection{Project Planning}

Our project planning has been based on the requirements, getting the core up and running first and from there, building everything around it. This has worked well four our part, since the team only consists of 6 people. If the case had been else vise, being a larger team, the planning would have had to plan the project even better. What needed to be done when was early documented, as shown in our Gannt diagram. For the report part the frames had already been set so this was an easier task to plan. There has always been something to do for everyone at any point, knowing this one was only to ask the team manager what should be done next.


\section{Development language}

The chosen development language language Javascript has been rewarding to work with. Being the most used language on the web, expanding our knowledge and skill with JavaScript has expanded our understanding of websites and programming in general. Compared to languages such as Java, which has strong typing, JavaScript is loosely typed, meaning the variable types are not explicitly stated and can be changed at runtime. It is also dynamic and has first-class function, which allows a lot of flexibility and means objects can be extended and modified at runtime, functions can be passed as arguments to other functions, and the program does not require compiling.\\
\newline
Although JavaScript introduces a lot of new, useful concepts, understanding how to use them in practice was a challenge at first. Examples of this include asynchronous functions, which are used a lot in Node.js and especially when communicating with a database. Understanding the flow of such programs required a lot of trial and error, but resulted in better understanding and more effective functions, compared to blocking functions for the same tasks.\\


\section{Customer interactions}

Having to deal with a customer for this project has been a great experience and opportunity to learn the dynamic of such a relationship. Meetings with the customer has happened almost weekly, this to show the newest progress and features in person. The discussions during the meetings have been about the new features and any objections the customer may have had. 

\section{Supervisor interactions}
Having a supervisor to lean on has been a great help when it comes to the report. Getting feedback throughout the course has been crucial to create the best possible documentations for the game. The supervisor has helped with questions about our development process and especially functional requirements. Meetings with the supervisor has happened every other week, the topic of discussion has been the problems the team was facing and the planned work for the next period. 

\section{Final product}

The game has been a major challenge, due to the large amount of requested features from the customer. Although the number of requested features was larger than what the team had time to implement, they were sorted by priority in talks with the customer, and it was never expected that all the features would be fully implemented. During the course of the project, the customer continued to request additional features, some of which were added on the fly, while others were put on a waiting list and not implemented due to time constraints.\\
\newline
The group is satisfied with the final product, having been able to implement many new features close to the end, and making the game as stable as was feasible. The customer expressed satisfaction with the product after testing it, even though the test might have been placed a bit too early when the game was not as stable as the final version. The final product functions as intended, even without the unfinished features.\\

\section{Lessons learned}


Early in the project there were daily meetings. Later on, the need for daily meetings was regarded as unnecessary as the team was able to work individually with the appointed tasks; therefore, the number of meetings was reduced to three days a week (Mondays, Wednesdays and Fridays). On Tuesdays and Thursdays the team members worked at home. Mondays were used as a sort of “kick-off”-day for the week and were used to plan specific tasks that were to be done during the week. This increased the efficiency.\\
\newline
During week 9 the group member responsible for sending status reports to the customer and supervisor was on vacation. Because of that, no status report was sent that week. The lesson learned from this is that communication within the team is crucial for the completion of all tasks in time.\\
\newline
In the start of the project, there was little communication within the group. No standards were set
for coding purposes and later on when the database names did not match the names in the program, there were some unnecessary problems. This was solved by setting code standards and better communication between the different parts of the team. \\
